%   \subsection{Environments}
%     
%     TeXStyle comes with a number of predefined environments that are intended 
%--------------------------------------------------
% Dependencies
  % Theorem styles
    \RequirePackage{amsthm}

  % Code listings
    \RequirePackage{listings}

  % Lists
    \RequirePackage[shortlabels]{enumitem}
%--------------------------------------------------

%--------------------------------------------------
% Theorem Styles
  \newtheoremstyle{NoBreak}
    {1pt}% measure of space to leave above the theorem. E.g.: 3pt
    {}% measure of space to leave below the theorem. E.g.: 3pt
    {\itshape}% name of font to use in the body of the theorem
    {\parindent}% measure of space to indent
    {}% name of head font
    {}% punctuation between head and body
    {.5em}% space after theorem head; " " = normal interword space
    {\thmname{#1}\thmnumber{ #2}:\thmnote{ #3}}% Manually specify head

  \newtheoremstyle{NoBreakNoNumber}
    {1pt}% measure of space to leave above the theorem. E.g.: 3pt
    {}% measure of space to leave below the theorem. E.g.: 3pt
    {\itshape}% name of font to use in the body of the theorem
    {\parindent}% measure of space to indent
    {}% name of head font
    {}% punctuation between head and body
    {.5em}% space after theorem head; " " = normal interword space
    {\thmname{#1}:\thmnote{ #3}}% Manually specify head

  \newtheoremstyle{Break}
    {5pt}% measure of space to leave above the theorem. E.g.: 3pt
    {5pt}% measure of space to leave below the theorem. E.g.: 3pt
    {\itshape}% name of font to use in the body of the theorem
    {\parindent}% measure of space to indent
    {}% name of head font
    {}% punctuation between head and body
    {\newline}% space after theorem head; " " = normal interword space
    {\thmname{#1}\thmnumber{ #2}:\thmnote{ #3}}% Manually specify head

  \newtheoremstyle{BreakNoNumber}
    {5pt}% measure of space to leave above the theorem. E.g.: 3pt
    {5pt}% measure of space to leave below the theorem. E.g.: 3pt
    {\itshape}% name of font to use in the body of the theorem
    {\parindent}% measure of space to indent
    {}% name of head font
    {}% punctuation between head and body
    {\newline}% space after theorem head; " " = normal interword space
    {\thmname{#1}:\thmnote{ #3}}% Manually specify head

  \newtheoremstyle{BreakBold}% name of the style to be used
    {5pt}% measure of space to leave above the theorem. E.g.: 3pt
    {5pt}% measure of space to leave below the theorem. E.g.: 3pt
    {\itshape}% name of font to use in the body of the theorem
    {\parindent}% measure of space to indent
    {\bfseries}% name of head font
    {}% punctuation between head and body
    {\newline}% space after theorem head; " " = normal interword space
    {\thmname{#1}\thmnumber{ #2}:\thmnote{ #3}}% Manually specify head

  \newtheoremstyle{BreakBoldNoNumber}% name of the style to be used
    {5pt}% measure of space to leave above the theorem. E.g.: 3pt
    {5pt}% measure of space to leave below the theorem. E.g.: 3pt
    {\itshape}% name of font to use in the body of the theorem
    {\parindent}% measure of space to indent
    {\bfseries}% name of head font
    {}% punctuation between head and body
    {\newline}% space after theorem head; " " = normal interword space
    {\thmname{#1}:\thmnote{ #3}}% Manually specify head

  \newtheoremstyle{NoBreakBold}% name of the style to be used
    {1pt}% measure of space to leave above the theorem. E.g.: 3pt
    {}% measure of space to leave below the theorem. E.g.: 3pt
    {\itshape}% name of font to use in the body of the theorem
    {\parindent}% measure of space to indent
    {\bfseries}% name of head font
    {}% punctuation between head and body
    {.5em}% space after theorem head; " " = normal interword space
    {\thmname{#1}\thmnumber{ #2}:\thmnote{ #3}}% Manually specify head

  \newtheoremstyle{NoBreakBoldNoNumber}% name of the style to be used
    {1pt}% measure of space to leave above the theorem. E.g.: 3pt
    {}% measure of space to leave below the theorem. E.g.: 3pt
    {\itshape}% name of font to use in the body of the theorem
    {\parindent}% measure of space to indent
    {\bfseries}% name of head font
    {}% punctuation between head and body
    {.5em}% space after theorem head; " " = normal interword space
    {\thmname{#1}:\thmnote{ #3}}% Manually specify head
%--------------------------------------------------

%--------------------------------------------------
% Environments

  \renewcommand{\qedsymbol}{$\lgblksquare$}

  % Theorems
    \theoremstyle{BreakBold}
    \newtheorem{theorem}{Theorem}[section]
    % Refreshes every theorem
    \theoremstyle{Break}
    \newtheorem{corollary}{Corollary}[theorem]
    \theoremstyle{Break}
    \newtheorem{lemma}{Lemma}[theorem]

  % Examples
    \theoremstyle{BreakBold}
    \newtheorem{examples}{Examples}[section]
    % Refreshes every example
    \theoremstyle{NoBreak}
    \newtheorem{example}{}[examples]

  % definitions
    \theoremstyle{BreakBold}
    \newtheorem{definition}{Definition}[section]
    % Refreshes every definition
    \theoremstyle{NoBreakBold}
    \newtheorem{subDef}{}[definition]

  % propositions
    \theoremstyle{BreakBold}
    \newtheorem{proposition}{Proposition}[section]
    % Refreshes every proposition
    \theoremstyle{NoBreak}
    \newtheorem{subProp}{}[proposition]

  % remarks/facts
    \theoremstyle{BreakNoNumber}
    \newtheorem{remarks}{Remarks}
    \theoremstyle{NoBreakNoNumber}
    \newtheorem{remark}{Remark}

  % Unumbered variants
    \theoremstyle{BreakNoNumber}
    \newtheorem*{theorem*}{Theorem}
    \newtheorem*{corollary*}{Corollary}
    \newtheorem*{lemma*}{Lemma}
    \newtheorem*{example*}{Example}
    \newtheorem*{definition*}{Definition}
    \newtheorem*{proposition*}{Proposition}
%--------------------------------------------------

%--------------------------------------------------
% Code Environments

  % Default
    \lstset{
      basicstyle=\fontsize{8}{10}\ttfamily,
      aboveskip={1.0\baselineskip},
      belowskip={1.0\baselineskip},
      columns=fixed,
      extendedchars=true,
      breaklines=true,
      tabsize=2,
      prebreak=\raisebox{0ex}[0ex][0ex]{\ensuremath{\hookleftarrow}}, % linebreak symbols
      frame=lines,
      showtabs=false,
      showspaces=false,
      showstringspaces=false,
      keywordstyle=\color[rgb]{0.627,0.126,0.941},
      commentstyle=\color[rgb]{0.133,0.545,0.133},
      stringstyle=\color[rgb]{01,0,0},
      numbers=left,
      numberstyle=\small,
      stepnumber=1,
      numbersep=10pt,
      captionpos=t,
      numbers=left,
      stepnumber=1,
      firstnumber=1,
      numberfirstline=true
    }

%--------------------------------------------------